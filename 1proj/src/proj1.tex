%%%%%
%Soubor: proj1.tex
%Datum: 28.02.2012
%Autor: Jan Wrona, xwrona00@stud.fit.vutbr.cz
%Projekt: Projekt c. 1 pro predmet ITY
%%%%%

\documentclass[a4paper, 10pt, twocolumn]{article}[26.02.2012]
  \usepackage[czech]{babel}
  \usepackage[utf8]{inputenc}
  \usepackage[T1]{fontenc}
  \usepackage[text={18cm, 25cm}, left=1.5cm, top=2.5cm]{geometry}
  \usepackage{url}

  \title{Typografická etiketa}
  \author{Vysázel: Jan Wrona\\xwrona00@stud.fit.vutbr.cz}
  \date{}

\begin{document}
\maketitle
%%%%%
\section{Hladká sazba}\label{hladka}
%%%%%
Hladká sazba používá jeden stupeň, druh a~řez písma a~je sázena na stanovenou šířku odstavce.
Je složena z~odstav\-ců, obvykle začínajících zarážkou, nejde-li o~první odstavec za nadpisem.
Mohou ale být sázeny i bez zarážky\,--\,rozho\-dující je celková grafická úprava.
Odstavec končí východovou řádkou. Každá věta začíná velkým písmenem, nesmí začínat číslicí.

Zvýraznění barvou, podtržením, ani změnou písma se v~odstavcích nepoužívá. Hladká sazba je
určena především pro delší texty, jako je beletrie. Porušení konzistence sazby působí
v~textu rušivě a~unavuje čtenářův zrak.
%%%%%
\section{Smíšená sazba}\label{smisena}
%%%%%
Smíšená sazba má volnější pravidla. Klasická hladká sazba se doplňuje o~další řezy
písma pro zvýraznění důležitých pojmů. Existuje \uv{pravidlo}:

\begin{quotation}
Čím více \texttt{druhů,} {\bf\emph{řezů,}} velikostí, barev písma {\fontfamily{cmss}\selectfont a~jiných efektů}
použi{\large jeme, }tím \emph{profesionálněji} {\Large bude} dokument vypadat. Čtenář {\tiny bude} {\bf nadšen!}
\end{quotation}

\textsc{Tímto pravidlem se nikdy nesmíme řídit.} Příliš časté zvýrazňování textových
elementů a~změny velikos\-ti písma jsou známkou amatérismu autora a~působí velmi rušivě.
Dobře navržený dokument nemá obsahovat více než 4 řezy či druhy písma. Dobře navržený
dokument je decentní, ne chaotický.

Důležitým znakem správně vysázeného dokumentu je konzistence\,--\,například {\bf tučný řez}
písma vyhradíme pou\-ze pro klíčová slova, \emph{skloněný řez} pouze pro doposud neznámé
pojmy a~nebudeme to míchat. Skloněný řez nepůsobí tak rušivě a~používá se častěji. V~\LaTeX u
jej sázíme raději příkazem \verb|\emph{text}| než \verb|\textit{text}|.

Smíšená sazba se nejčastěji používá pro sazbu vědeckých článků a~technických zpráv.
U~delších dokumentů vědeckého či technického charakteru je zvykem vysvětlit význam
různých typů zvýraznění v~úvodní kapitole.
%%%%%
\section{Další rady:}\label{rady}
%%%%%
\begin{itemize}
  \item Nadpis nesmí končit dvojtečkou a~nesmí obsahovat odkazy na obrázky, citace, poznámky pod čarou,\,\dots
  \item Nadpisy, číslování a~odkazy na číslované elementy mu\-sí být sázeny příkazy k~tomu
určenými. Maximálně využíváme možností \LaTeX u a~zvolené třídy dokumen\-tu.
  \item Výčet ani obrázek nesmí začínat hned pod nadpisem a~nesmí tvořit celou kapitolu.
  \item Poznámky pod čarou\footnote{Příliš mnoho poznámek pod čarou čtenáře zbytečně rozptyluje.}
používejte opravdu střídmě. (Šetřete i s~textem v~závorkách.)
  \item Nepoužívejte velké množství malých obrázků. Zvažte, zda je nelze seskupit.
  \item Bezchybným pravopisem a~sazbou dáváme najevo úctu ke čtenáři. Odbytý
text s~chybami bude čtenář právem považovat za nedůvěryhodný.
\end{itemize}
%%%%%
\section{České odlišnosti}\label{ceske}
%%%%%
Česká sazba se oproti okolnímu světu v~některých aspektech mírně liší. Jednou
z~odlišností je sazba uvozovek. Uvozovky se v~češtině používají převážně pro zobrazení
přímé řeči, zvýraznění přezdívek a~ironie. V~češtině se používají uvozovky typu
\uv{9966} místo anglických ``uvozovek'' nebo dvojitých "uvozovek". Lze je sázet
připravenými příkazy nebo při použití UTF-8 kódování i přímo. Obě možnosti mají své
výhody i úskalí.

Ve smíšené sazbě se řez uvozovek řídí řezem prvního uvozovaného slova. Pokud je
uvozována celá věta, sází se koncová tečka před uvozovku, pokud se uvozuje slovo
nebo část věty, sází se tečka za uvozovku.

Druhou odlišností je pravidlo pro sázení konců řádků. V~české sazbě by řádek neměl
končit osamocenou jednopísmennou předložkou nebo spojkou. Spojkou \uv{a} končit může
pouze při sazbě do šířky 25 liter. Abychom \LaTeX u zabránili v~sázení osamocených
předložek, spojujeme je s~následujícím slovem nezlomitelnou mezerou. Tu sázíme pomocí
znaku \verb|~| (vlnka, tilda). Pro systematické doplnění vlnek slouží volně šiřitelný
program \emph{vlna} od pana Olšáka\footnote{Viz \url{ftp://math.feld.cvut.cz/pub/olsak/vlna/}.}.

Principiálně lepší řešení nabízí balík \emph{encxvlna}, od pánů Olšáka
a~Wagnera\footnote{Viz \url{http://tug.ctan.org/pkg/encxvlna}.}.
Pro jeho použití je ovšem potřeba speciální konfigurace \LaTeX u.
%%%%%
\section{Závěr}\label{zaver}
%%%%%
Jistě jste postřehli, že tento dokument obsahuje schválně několik typografických prohřešků.
Sekce \ref{smisena} a~\ref{rady} obsahují typografické chyby. V~kontextu celého textu je jistě snadno najdete.
Je dobré znát možnosti \LaTeX u, ale je také nutné vědět, kdy je nepoužít.
\end{document}
