%%%%%
%Soubor: proj2.tex
%Datum: 18.03.2012
%Autor: Jan Wrona, xwrona00@stud.fit.vutbr.cz
%Projekt: Projekt c. 2 pro predmet ITY
%%%%%

\documentclass[a4paper, 11pt, twocolumn]{article}[18.03.2012]
  \usepackage[czech]{babel}
  \usepackage[utf8]{inputenc}
  \usepackage[T1]{fontenc}
  \usepackage[text={18cm, 25cm}, left=1.5cm, top=2.5cm]{geometry}
  \usepackage{amsfonts}
  \usepackage{amsmath}
  \usepackage{amsthm}
  \usepackage{mdwlist}


\newtheorem{defi}{Definice}[section]
\newtheorem{algo}[defi]{Algoritmus}
\newtheorem{veta}{Věta}

\begin{document}
%%%%%
%Soubor: title.tex
%Datum: 18.03.2012
%Autor: Jan Wrona, xwrona00@stud.fit.vutbr.cz
%Projekt: Projekt c. 2 pro predmet ITY
%%%%%
\begin{titlepage}
\begin{center}
\textsc{{\LARGE Vysoké učení technické v Brně}\\
\smallskip
{\Large Fakulta informačních technologií}}\\
\vspace{\stretch{0.382}}
{\LARGE Typografie a publikování\,--\,2. projekt}\\
\medskip
{\Huge Sazba dokumentů s matematickými výrazy}\\
\vspace{\stretch{0.618}}
\end{center}
{\Large Jan Wrona \hfill \today}
\end{titlepage}

%%%%%
\section{Úvod}
%%%%%
Tato úloha je zaměřena na sazbu titulní strany a~textů, které obsahují
matematické vzorce, rovnice (\ref{rovnice1}), prostředí (například definice \ref{def:bezkon}
na straně \pageref{def:bezkon}).

Na titulní straně je využito sázení nadpisu podle optického středu s~využitím
\emph{zlatého řezu}. Tento postup byl probírán na přednášce. Pro sazbu matematických
elementů byly využity balíky \AmS-\LaTeX u.
%%%%%
\section{Plynulý matematický text}
%%%%%
Zásady pro sazbu matematiky v~plynulém textu odpovídají zásadám pro smíšenou
sazbu.

Pro množinu $M$ označuje $\mathrm{card}(M)$ kardinalitu $M$. Pro množinu $M$ reprezentuje $M^*$
volný monoid generovaný množinou $M$ s~operací konkatenace. Prvek identity ve
volném monoidu $M^*$ značíme symbolem $\varepsilon$. Nechť $M^+=M^*-\{\varepsilon\}$.
Algebraicky je tedy $M^+$ volná pologrupa generovaná množinou $M$ s~operací \-konkatenace. Konečnou
neprázdnou množinu $M$ nazvěme $abeceda$. Pro $w\in M^*$ označuje $|w|$ délku řetězce $w$.
Pro $W\subseteq M$ označuje $\mathrm{occur}(w,W)$ počet výskytů symbolů z~$W$ v~řetězci $w$
a~$\mathrm{sym}(w,i)$ určuje $i$-tý symbol řetězce $w$; například $\mathrm{sym}(abcd,3)=c$.
%%%%%
\section{Sazba definic a~vět}
%%%%%
Pro sazbu definic a~vět slouží balík \texttt{amsthm}.

\begin{defi} \label{def:bezkon}
\emph{Bezkontextová gramatika} je čtveřice $G=(V,T,P,S)$, kde
\begin{description*}
  \item[$V$] je totální abeceda,
  \item[$T\subseteq V$] je abeceda terminálů,
  \item[$S\in (V-T)$] je startující symbol,
  \item[$P$] je konečná množina \emph{pravidel} tvaru $q\colon A\rightarrow \alpha$, kde $A\in (V-T)$, $\alpha \in V^*$
a~$q$ je návěští tohoto
pravidla.
\end{description*}
Nechť $N=V-T$ značí abecedu neterminálů. Pokud $q\colon A\rightarrow \alpha \in P$, $\gamma,\delta \in V^*$, $G$
provádí derivační krok z~$\gamma A\delta$ do $\gamma \alpha \delta$ podle pravidla
$q\colon A~\rightarrow \alpha$, symbolicky píšeme $\gamma A\delta \Rightarrow \gamma \alpha \delta\ [q\colon A~\rightarrow \alpha]$
nebo zjednodušeně $\gamma A\delta \Rightarrow \gamma \alpha \delta$.
Standardním způsobem definujeme $\Rightarrow^n$, kde $n\geq 0$. Dále definujeme tranzitivní uzávěr
$\Rightarrow^+$ a~tranzitivně-reflexivní uzávěr $\Rightarrow^*$.
\end{defi}

Algoritmus můžeme uvádět textově, podobně jako definice, nebo lze použít
pseudokódu vysázeného ve vhodném prostředí (například \texttt{algorithm2e}).

\begin{algo}
\emph{Ověření bezkontextovosti gramatiky.} Mějme gramatiku $G=(N,T,P,S)$.
\begin{enumerate}
  \item Pro každé pravidlo $p\in P$ proveď test, zda $p$ na levé straně obsahuje právě
jeden symbol z~$N$.\label{zaprve}
  \item Pokud všechna pravidla splňují podmínku z~kroku \ref{zaprve}, tak je gramatika $G$
bezkontextová.
\end{enumerate}
\end{algo}

\begin{defi}
Jazyk definovaný gramatikou $G$ definujeme jako $L(G)=\{w\in T^*\;|\;S\Rightarrow^* w\}$.
\end{defi}

\subsection{Podsekce obsahující větu}
Věty a~definice mohou mít vzájemně nezávislé číslování. Důkaz se obvykle uvádí
hned za větou.

\begin{defi}
Nechť $L$ je libovolný jazyk. $L$ \emph{je bezkontextový jazyk},
když a~jen když $L=L(G)$, kde $G$ je libovolná bezkontextová gramatika.
\end{defi}
\begin{defi}
Množinu $\mathcal{L}_{CF}=\{L|$L je bezkontextový jazyk$\}$
nazýváme třídou bezkontextových jazyků.
\end{defi}
\begin{veta} \label{vetajedna}
Nechť $L_{abc}=\{a^nb^nc^n|n\geq 0\}$. Platí, že $L_{abc}\notin \mathcal{L}_{CF}$.
\end{veta}
\begin{proof}
Důkaz se provede pomocí Pumping lemma pro bezkontextové jazyky a~je
zřejmý, což implikuje pravdivost věty \ref{vetajedna}.
\end{proof}
%%%%%
\section{Rovnice a~odkazy}
%%%%%
Složitější matematické formulace sázíme mimo plynulý text. Lze umístit několik
výrazů na jeden řádek, ale pak je třeba tyto vhodně oddělit, například příkazem
\verb|\quad|.
\[\sqrt[a^8]{_4^3b^2_1}\quad \mathbb{N}=\{1,2,3,\dotsc\}\quad x^{y^y}\neq x^{yy}\quad z_{i_j}\not\equiv z_{ij}\]

V~rovnici (\ref{rovnice1}) jsou využity tři typy závorek s~různou explicitně
definovanou velikostí.
\begin{align}
x\ &=\ -\bigg(\Big\{[a*b]^c-d\Big\}+1\bigg)\label{rovnice1}\\
s\ &=\ \sqrt{\frac{1}{n}\sum\limits_{i=1}^s p_i (x_i-x)^2}\nonumber
\end{align}

V~této větě vidíme, jak vypadá implicitní vysázení limity $lim_{n\to \infty}\,f(n)$ v~normálním
odstavci textu. Podobně je to i s~dalšími symboly jako $\sum_1^n$ či $\bigcup_{A\in \mathcal{B}}$.
V~případě vzorce $\lim\limits_{x\to 0}\frac{\sin x}{x}=1$ jsme si vynutili méně úspornou sazbu příkazem \verb|\limits|.
\begin{align}
\int_a^b f(x)\,\mathrm{d}x\ &=\ -\int\limits_b^a f(x)\,\mathrm{d}x\\
\overline{\overline{A}\wedge\overline{B}}\ &=\ \overline{\overline{A\vee B}}
\end{align}
%%%%%
\section{Složené zlomky}
%%%%%
Při sázení složených zlomků dochází ke zmenšování použitého písma v~čitateli
a~jmenovateli. Toto chování není vždy žádoucí, protože některé zlomky potom mohou
být obtížně čitelné.

V~těchto případech je možné nastavit standardní stupeň písma v~podvýrazech
ručně pomocí příkazu \verb|\displaystyle| u~vysázených vzorců nebo pomocí \verb|\textstyle|
u~vzorců, které jsou součástí textu. Srovnejte:
\[\dfrac{\dfrac{(a+b)^2}{x+y}-\dfrac{x-y}{\dfrac{a}{b}}}{1-\dfrac{a+b}{a-b}}\quad
\frac{\frac{(a+b)^2}{x+y}-\frac{x-y}{\frac{a}{b}}}{1-\frac{a+b}{a-b}}\]
Tento postup lze použít nejen u~zlomků.
\[\prod_{i=0}^{m-1} (n-i) = \overbrace{n(n-1)(n-2)\dots (n-m+1)}^{\displaystyle{m \text{ je počet činitelů}}}\]
%%%%%
\section{Matice}
%%%%%
Pro sázení matic se velmi často používá prostředí array a~závorky (\verb|\left|,
\verb|\right|). Tyto příkazy vždy tvoří pár a~nelze je použít samostatně.
\begin{align*}
& \begin{pmatrix}
a+b & a-b\\
\widetilde{c+d} & \tilde{b}\\
\vec{a} & \underleftrightarrow{AC}\\
\xi & \aleph
\end{pmatrix}\\
A=
& \begin{Vmatrix}
a_{11} & a_{12} & \cdots & a_{1n}\\
a_{21} & a_{22} & \cdots & a_{2n}\\
\vdots & \vdots & \ddots & \vdots \\
a_{m1} & a_{m2} & \cdots & a_{mn}
\end{Vmatrix}\\
& \begin{vmatrix}
h & i \\
w & x
\end{vmatrix}
= hx-iw
\end{align*}
Prostředí \verb|array| lze úspěšně využít i jinde.
\[
\binom{n}{k} = \left\{
\begin{array}{l l}
0 & \quad \text{pro $k<0$ nebo $k>n$}\\
\frac{n!}{k!(n-k)!} & \quad \text{pro $0\leq k\leq n$}
\end{array}
\right.
\]
%%%%%
\section{Závěrem}
%%%%%
V~případě, že budete potřebovat vyjádřit matema\-tickou konstrukci nebo symbol
a~nebude se Vám dařit jej nalézt v~samotném \LaTeX u, doporučuji prostudovat
možnosti balíku maker \AmS-\LaTeX. Analogická poučka platí obecně pro jakoukoli
konstrukci v~\TeX u.
\end{document}
