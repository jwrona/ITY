%%%%%
%Soubor: proj4.tex
%Datum: 15.04.2012
%Autor: Jan Wrona, xwrona00@stud.fit.vutbr.cz
%Projekt: Projekt c. 4 pro predmet ITY
%%%%%

\documentclass[a4paper, 12pt]{article}[15.04.2012]
  \usepackage[czech]{babel}
  \usepackage[utf8]{inputenc}
  \usepackage[T1]{fontenc}
  \usepackage{palatino}
  \usepackage[text={17cm, 24cm}, left=2cm, top=3cm]{geometry}
  \usepackage{url}

\begin{document}
%%%%%
%Soubor: title.tex
%Datum: 15.04.2012
%Autor: Jan Wrona, xwrona00@stud.fit.vutbr.cz
%Projekt: Projekt c. 4 pro predmet ITY
%%%%%
\begin{titlepage}
\begin{center}
\textsc{{\Huge Vysoké učení technické v Brně}\\
\medskip
{\huge Fakulta informačních technologií}}\\
\vspace{\stretch{0.382}}
{\LARGE Typografie a publikování\,--\,4. projekt}\\
\medskip
{\Huge Citace}\\
\vspace{\stretch{0.618}}
\end{center}
{\Large \today \hfill Jan Wrona}
\end{titlepage}


Knižní písmo, označované také jako font \cite{software08}, tak jak ho známe dnes, má za sebou velmi
dlouhou historii \cite{mocicka05}. Zameříme-li se na latinku, zjistíme, že její kořeny sahají do
starověkého Řecka \cite{wiki-latinka}. Již v~tomto raném stádiu
vývoje latinky, můžeme pozorovat větvení na formy písma: především to byly kapitála a starší římská kurzíva.
Z~prvně jmenované se vyvinuly velká písmena, tedy verzálky. Postupem času prošlo toto písmo vývojem,
vznikl nespočet variací, řezů a typů tohoto písma \cite{computers05}. Ovšem ne všechny jsou v~dnešní době běžně používaná, což
může být způsobeno například jejich nevhodností pro počítačobou sazbu \cite{bednar11}\cite{zongker00}.

Běžná kinžní písma můžeme podle jejich stavby rozdělit do čtyř kategorií \cite{rybicka03}:
\begin{description}
  \item[Serifová] Též antikvová písma, jsou význačné především svými patkami (tzv. serify). Ty přispívají
  k~dobré čitelnosti těchto písem \cite{janak01}, protože patky tvoří linii a lépe vedou oči po řádku. Druhou vlastností
  tohoto rodu písem jsou stíny. Nejsou to ale stíny jak je známe, ale tyto stíny značí, že jednotlivé tahy
  nemusí být stejně široké. Typickým zástupcem v~počítačové sazbě je font Times.
  \item[Bezserifová] Jak již název napovídá, tyto písma se liší především absencí patek. Jejich čitelnost
  ovšem oproti serifovým písmům mírně pokulhává \cite{goossens94}, jsou ale výraznější. Zdejším typickým zástupcem je
  Helvetica či Arial.
  \item[Kombinace] Další kategorií jsou písma kombinující dvě předchozí kategorie. Mohou to být tzv. egyptienky (patky, bez stínů)
  nebo groteskantikvy (bez patek, stíny) \cite{antypa-pismo}.
  \item[Ostatní] Tato kategorie zahrnuje předevšíp písma, které se nedají zařadit do žádné z~předchozích kategorií.
  Patří sem různá zdobená písma, nebo například písma nabodobující lidský rukopis (kaligrafická).
\end{description}

\bibliographystyle{plain}
\bibliography{proj4}

\end{document}
